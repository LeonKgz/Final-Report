\chapter{Introduction}

% According to Cisco IBSG the Internet of Things was born somwhere between 2008 and 2009 at “the point in time when more ‘things or objects’ were connected to the Internet than people”\footnote{https://www.cisco.com/c/dam/en_us/about/ac79/docs/innov/IoT_IBSG_0411FINAL.pdf}\footnote{https://www.cisco.com/web/offer/emear/29676/tnc2013/documents/IoEandOlympics.pdf}. Furthermore in 2020 for the first time there are more "IoT connections" (e.g., connected cars, smart home devices, connected industrial equipment) than there are "non IoT connections" (smartphones, laptops, and computers) according to the latest report by  IOT Analytics\footnote{https://iot-analytics.com/state-of-the-iot-2020-12-billion-iot-connections-surpassing-non-iot-for-the-first-time/}. The paradigm is becoming evermore relevant and with the global pandemic it's focus would expand from using smart sensors for saving energy to social distancing in the workplace and tracking technologies to fight COVID-19\footnote{https://www.iotworldtoday.com/2021/01/07/iot-trends-2021-a-focus-on-fundamentals-not-nice-to-haves/}.\\

Internet of Things (IoT) devices typically operate on strict energy constraints having limited power (battery or solar) and communicating over long distances. In contrast to the existing 4th or 5th generation communication technologies, the high data rate is typically traded for longer communication ranges. This can be achieved with Low-Power Wide-Area Networks (LPWAN). Among the various LPWAN technologies, this project will focus on the LoRa (Long Range) modulation scheme.\\
% \footnote{https://www.forbes.com/companies/semtech/?sh=36d0faf84e3e}.\\

The main focus of this project will be to look into ways to apply different reinforcement learning techniques for 
effective resource management of IoT devices to maximize efficiency, reliability and coverage of LoRaWAN protocol for LPWAN with a focus on Nonorthogonal Multiple Access (NOMA) signal decoding technique. \\

In chapter 2, I will look into basic concepts of LoRa and LoRaWAN protocol. In addition, I will review the simulation framework I'll be using. In chapter 3, I will review the Reinforcement Learning 
background and potential approaches applicable in this setting.
Chapter 5 will outline my contributions and modifications
to the original framework along with the implementation of the learning model. Chapter 6 will focus on the performance of the proposed model and its evaluation.
